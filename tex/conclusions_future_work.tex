\chapter{Conclusions and Future Work} \label{chap:conclusions}
 
% Purpose and motivations
The work presented in this thesis was driven by the desire to create interactive art sculptures that possess the characteristics of living things \cite{Gorbet2015}. In previous generations of the Hylozoic Series interactive art sculptures \cite{Beesley2012}, complex behaviours emerged from the superpositions of a set of simple prescripted responsive behaviours. In this thesis, we introduced an autonomous behaviour generation system by applying a reinforcement learning algorithm that mimics curiosity. In addition, we examined the interactions between the CBLA system and the human users and how it compared with prescripted behaviours.
 
\section{CBLA System}
% CBLA with multiple agents
The CBLA builds on the Intrinsic Adaptive Curiosity (IAC) algorithm \cite{Oudeyer2007} created for developmental robots. CBLA enables the sculptural system to select actions that lead to maximum potential knowledge gains. In order to implement the CBLA on a distributed sculptural system with hundreds of sensors and actuators, a novel formulation that uses a network of CBLA Nodes, with each representing a subset of the sculptural system, was developed. Each CBLA Node runs its own copy of the CBLA and constructs prediction models that model the relationships between its states and actions, and the consequences. The multi-Nodes approach enables each Node to operate at a rate that is appropriate for its associated actuators and sensors. In addition, this reduces the dimension of the state space which in turn reduces the number of samples required to acquire the model. These Nodes are then connected via shared inputs and virtual inputs. These links allow information from one Node to propagate to other parts of the system. 

% Idling action
Furthermore, a mechanism that relates the maximum activation levels based on the relative action value of the system was developed to give a visual indication of the CBLA Node's appraisal of the knowledge gain potentials in its associated regions. 
% Examined behaviours under different configurations
The behaviours of the system under different network configurations and action selection policies when an external trigger is applied were examined. Using the Spatial Mode network configuration, which connects Nodes based on their spatial proximities, enables the CBLA system to respond with a higher level of activation near the source of the trigger immediately after the triggering event. In addition, using a high sensitivity setting enables a higher level of activation to both ambient noise and deliberate triggers. These settings were selected for the user study conducted in this thesis.

In future work, experiments should be conducted using different parameters and settings to examine their effects on the behaviour of the system. Parameters such as the sensitivity setting, the loop period of each CBLA Node, and the thresholds of the split criteria; and settings such as the network configuration, region splitting mechanism, action selection policy, and type of the prediction models may all have profound effects on the behaviours of the system. For instance, while a smaller loop period may shorten the response time, it will require that new action selection policies or controllers for the actuators to be developed. While using Lasso has the advantage of reducing the dimensionality of the model, it may inadvertently decouple the links between CBLA Nodes. 

In addition, even though mechanisms were put in place to control the number of regions in a CBLA Engine, it would still continue to grow indefinitely under the current implementation. New mechanisms, such as merging similar regions, and removing obsolete regions, should be investigated and developed to facilitate long term operation of the CBLA system.


\section{Interactive Control System}
% Built a test bed that allows us to observe the behaviours
Implementation of a CBLA system requires the control and sampling of a larger number of sensors and actuators and at a higher frequency than what the previous generations of the interactive control system could handle. Therefore, a new set of electronic hardware and control software was developed. This new interactive control system enables the interfacing of hundreds of actuators and sensors and the capability of high speed communication with a computer through USB. This simplified the implementation of a computationally intensive and multi-threaded CBLA system by running it on a standard computer instead of an embedded platform. A four-cluster experimental test bed was built for the experiments involving human users.

In future work, the interactive control system should be implemented on a distributed platform. This reduces the risk of system-wide failure caused by the problems at the central computer. In addition, it would enable the system to support an even larger number of Nodes that require more computational power and a larger number of connection ports than are available to a single computer. Furthermore, a more robust database and data logging module should be developed to allow long term data collection and real-time data retrieval. 


\section{User Interaction}
% CBLA is less interesting than Prescripted Mode
A study that involved the users interacting with the sculpture one at time was conducted. During the interaction, the users reported their interest levels periodically. By drawing connections between the users' responses and the type of behaviours, it was shown that when the sculpture was switched from CBLA Mode to Prescripted Mode, the users' levels of interest increased. This means that under the circumstances of this user study, behaviours generated by the CBLA system were less interesting than the prescripted behaviours. 
% proximity and type of activation has no effect
Furthermore, the proximity of the activations did not seem to influence the users' level of interest. We also did not find a particular type of Node to be more interesting than the overall system. However, though unclear why, we did find the activation levels of the Reflex Node to be uncorrelated to the users' levels of interest. 

% relationship between activations, responsiveness, and interest level
Weak to moderate positive correlations were found between activation levels, perceived responsiveness of the system, and the users' levels of interest. This suggests that systems with activations that are more prominent and with more comprehensible relationships with the users' actions may engage the users better, at least during the short term interaction investigated in this study. The less predictable behaviours generated by a CBLA system are perhaps one of the reasons why its behaviours was reported to be less interesting than the simple and highly responsive prescripted behaviours. 

% == Future work ===
In future work, the effects of response time and predictability of the sculptural system's responsive behaviour on the users' perceived responsiveness should be further investigated. More experiments should be conducted to model the response time of a CBLA Node to different kinds of triggers, under different configurations. In addition, the repeatability of the behaviours should be quantified to enable comparisons of the users' perceptions under different levels of predictability. After all, if an user cannot predict the response of the sculpture repeatedly, he or she would have difficulty to associate the responsive behaviours to his or her actions.

Although a positive correlation was found between activation levels and the users' level of interest, there is no evidence of causation. It is possible that users who were more interested in the sculpture, due to other reasons, interacted with the sculpture more and caused higher levels of activation. In future work, the hypothesis that activations promote higher user engagement should be tested by studying the responses of the users to a system that ignores all sensory inputs and activates at different level of intensity autonomously. The results from that study would improve our understanding of the importance of the interactive aspect relative to the performative aspect of system.

% people have different expectation
Through observations and the responses from the exit questionnaires, we observed that different participants in the study interacted with the sculpture in very different ways, and had very different expectations about their experience. This suggests that, in future work, more meaningful correlations may be revealed by categorizing the different types of users. Alternatively, to align participants' expectations, the expected behaviours and meanings behind the concepts of the sculpture can be explained to the participants prior to the study. This is similar to how visitors may read and learn about the sculptures prior to interacting with them in a museum setting. Therefore, in future work, studies should be conducted to investigate the effects of prior knowledge on the users' appraisal of their experience. 

% big differences between the test bed and the actual sculpture
In addition, though the test bed resembled a typical interactive art sculpture, there were also significant differences. For instance, the small size of the test bed means that the proximity of the activation became less of a factor since all Nodes were close to the user. In addition, in a public exhibition, users' interactions with a sculpture might be secondary to their primary activities such as socializing with friends, or simply passing by to get from one location to another. Users may find the experience more interesting as an augmentation to their primary activities in comparison to the more focused, one-on-one experience tested in this thesis. Therefore, in future work, user studies that better reflect the actual use cases should be designed and conducted to further examine the users' experience in interacting with the sculpture. 

Indeed, the advantage of using CBLA, which generates complex and changing behaviours, may only be apparent in a longer study. From the written responses of the questionnaires (\Cref{Append:User Study Written Responses}), the participants had a tendency to want to learn to ``play'' the system and understand the its interactive behaviours. Therefore, simple and easily predictable behaviours may had been preferable. However, we hypothesize that as novice users become experts, those simple behaviours may become boring, while the more complex and evolving behaviours of a CBLA system may become more interesting. In future work, longer studies on the same users that span over a few days, or even months, should be conducted to test this hypothesis.

