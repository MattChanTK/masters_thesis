\chapter{Conclusions and Future Work} \label{chap:conclusions}
 
% Purpose and motivations
The work presented in this thesis was driven by the desire to create interactive art sculptures that possess the characteristics of living things \cite{Gorbet2015}. In previous generations of the Hylozoic Series interactive art sculpture \cite{Beesley2012}, complex behaviours emerged from the superpositions of a set of simple prescripted responsive behaviours. In this thesis, we introduced elements of a living system such as the adaptability, self-motivated growth, and autonomy by applying a reinforcement learning algorithm that mimics curiosity. In addition, we examined the interactions between the CBLA system and the human users and how it compared with prescripted types of behaviours.
 
% CBLA with multiple agents
The CBLA builds on the Intrinsic Adaptive Curiosity (IAC) \cite{Oudeyer2007} that was created for developmental robots. CBLA enables the sculptural system to select actions that lead to maximum potential knowledge gains. In order to implement the CBLA on a distributed sculptural system with hundreds of sensors and actuators, a novel formulation that uses a network of CBLA Nodes, with each representing a subset of the sculptural system, was developed. Each CBLA Node runs its own copy of the CBLA and constructs prediction models that model the relationships between its states and actions, and the consequences. The multi-Nodes approach enables each Node to operate at a rate that is appropriate for its associated actuators and sensors. In addition, this reduces the dimension of the state space which in turn reduces the number of samples required to acquire the model. These Nodes are then connected via shared inputs and virtual inputs. These links allows information from one Node to propagate to other parts of the system. Furthermore, a mechanism that relates the maximum activation levels based on the relative action value of the system was developed to give visual indications of the CBLA Node's appraisal of the knowledge gain potentials in its associated regions. The behaviours of the system under different network configurations and action selection policies when an external trigger is applied were examined. Using Spatial Mode network configuration, which connects Nodes based on their spatial proximities, enables the CBLA system to respond with a higher level of activations near the source of trigger in the 30 seconds after the triggering event. In addition, using a high sensitivity setting enables a higher level of activations to both ambient noise and deliberate triggers. These settings were selected for the user study conducted in this thesis.

% Built a test bed that allows us to observe the behaviours
Implementation of a CBLA system requires the control and sampling of a larger number of sensors and actuator and at a higher frequency than what the previous generations of the interactive control system can handle. Therefore, a new set of electronic hardware and control software was developed. This new interactive control system enables the interfacing of hundreds of actuators and sensors and the capability of high speed communication with a computer through USB. This simplified the implementation of a computationally intensive and multi-threaded CBLA system by running it on a standard computer instead of an embedded platform. A four-cluster experimental test bed was built for the experiments involving human users.

% Ran user study, reveal some thing
A study that involved the users interacting with the sculpture one at time was conducted. During the interaction, the users reported their interest levels periodically. By drawing connections between the users' responses and the type of behaviours, it was shown that when the sculpture was switched from CBLA Mode to Prescripted Mode, the users' levels of interest increased. This means that under the circumstances of this user study, behaviours emerging from the CBLA system were in fact less interesting than the prescripted behaviours. On the other hand, weak to moderate positive correlations were found among activation levels, perceived responsiveness of the system, and the users' levels of interest. This suggests that systems with activations that are more prominent and with more comprehensible relationships with the users' actions may engage the users better. The subtlety and complexity of the behaviours emerging from a CBLA system are perhaps one of the reasons why its behaviours was reported to be less interesting than the simple though highly responsive prescripted behaviours. While this might be the case, it is also interesting to contemplate whether the purpose of a living system is indeed to entertain or engage people. Furthermore, proximity of the activations did not seem to influence the users' level of interest. We also did not find a particular type of Nodes to be more interesting than the overall system. However, though unclear why, we did find the activation levels of the Reflex Node to be uncorrelated to the users' levels of interest. 

% Future work
On the other hand, through observations and the responses from the exit questionnaires, different participants in the study interacted with the sculpture in very different ways and they had very different expectations about their experience. This suggests that more meaningful correlations may be revealed by categorizing the different types of users. Alternatively, to align participants' expectations, the expected behaviours and meanings behind the concepts of the sculpture can be explained to the participants prior to the study. This is similar to how visitors may read and learn about the sculptures prior to interacting with them. 
 
In addition, though the test bed resembled a typical interactive art sculpture, there were also significant differences. For instance, the small size of the sculptures means that the proximity of the activation became less of a factor since all Nodes may be considered as being close to the users. In addition, in a public exhibition, users' interaction with sculpture might be secondary to their primary activities such as socializing with friends, or simply passing by to get from one location to another. Users may find the experience more interesting as an augmentation to their primary activities in comparison to the more focused, one-on-one experience tested in this thesis. Therefore, in future work, user studies that better reflect the actual use cases should be designed and conducted to further examine the users' experience in interacting with the sculpture.
