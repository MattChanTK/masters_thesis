\chapter[Introduction]{Introduction\footnote{Part of this chapter is adapted from papers published in IROS~2015~\cite{Chan2015} and Next Generation \mbox{Building}~\cite{Gorbet2015}}} 
\label{chap:intro}

% The context is interactive arts sculpture made by PBAI
Interactive arts are a type of art form that requires the involvement of the spectators to achieve its purpose. With recent advances in capabilities and miniaturization of computers, sensors and actuators, artists now have access to more tools to create highly complex interactive artworks. In the Hylozoic Series of kinetic sculptures built by Philip Beesley Architect Inc.(PBAI)\footnote{Philip Beesley Architect Inc.: \url{http://philipbeesleyarchitect.com}}, the designers use a network of microcontrollers to control and sample hundreds of actuators and sensors \cite{Beesley2010}\cite{Beesley2010-1}. Each node in the network can perform a simple set of interactive behaviours. While the behaviours have previously been either prescripted or random, complex group behaviours have been seen to emerge through communication among nodes and interaction with the spectators \cite{Beesley2012}. \Cref{fig:other-sculpture-photo} shows a Hylozic Series interactive sculpture that was installed in the Museum of Modern and Contemporary Art in Seoul, South Korea \cite{PBAISeoul2013}. 

\begin{figure} [!htb]
	\centering
	\includegraphics[width=1.0 \textwidth]{"fig/introduction/PBAI_09"}
	\caption[Photograph of a Hylozoic Series interactive art sculpture, Epiphyte Chamber]{Photograph of a Hylozoic Series interactive art sculpture, Epiphyte Chamber, installed in the Museum of Modern and Contemporary Art in Seoul, South Korea in 2013 (provided by the Philip Beesley Architect Inc.).}
	\label{fig:other-sculpture-photo}
\end{figure}

% Expand first on the goal of interactive architecture and to what extent they are met/not met by existing systems.
One of the goals of the Hylozoic Series interactive art sculpture is to invite the users to contemplate if architecture can become alive, albeit in very primitive ways \cite{Beesley2012}. Indeed, the name comes from ancient Greek philosophy, \textit{hylozoism}, which is a belief that all matter is alive in some sense. While previous generations of the sculptural systems can exhibit some primitive responsive behaviours, they are nevertheless manually designed by its designers and remain unchanged over time. To take on the characteristics of a higher level living system, the sculptural system should have the ability to generate and modify its own behaviours, and adapt to changes in external environment.

% Explain why is it a problem that interactive behaviors remain unchanged over time?
Practically, as a public interactive art piece, it should also be engaging and interesting to the users. We hypothesize that, if the behaviours do not change, users would find the sculpture less interesting over time as its behaviours become predictable to the users. This is undesirable for a permanent installation in which the same users may interact with the sculpture over an extended period of time. 
% Explain why is that fact that behaviors are designed manually a problem?
Moreover, in a system with a large number of sensors and actuators, programming a complex set of carefully choreographed behaviours is complicated and requires lengthy implementation and testing processes. Furthermore, manually designing the behaviours requires the human designers to predict behaviours that could induce positive user reactions, which is very subjective. 

% Introduce the approach taken in the thesis
To address the challenge of long-term, adaptive engagement, and self-motivated autonomy, a new Hylozoic Series was developed. We introduced the Curiosity-Based Learning Algorithm (CBLA) to the Hylozoic series installation. The CBLA re-casts the interactive sculpture as a set of agents driven by an intrinsic desire to learn. Presented with a set of input and output variables that it can observe and control, each agent tries to understand its own mechanisms, its surrounding environment, and the occupants, by learning models relating its inputs and outputs. We hypothesize that the occupants will find the behaviours which emerge during CBLA-based control to be interesting, more life-like, and less robotic than ones emerged from prescripted behaviours that are manually designed. 

% What are we doing in this work
In this thesis, we implemented the physical manifestation of a learning algorithm in an abstract interactive art sculpture. Working with PBAI, a new series of interactive sculptures was designed and built to generate its own behaviours through learning and interaction with its physical environment. We observed how occupants of the space perceive and respond to the behaviours of the sculpture. 

The learning algorithm was based on the Intelligent Adaptive Curiosity \cite{Oudeyer2007} (IAC) algorithm developed by Oudeyer et al. It is a reinforcement learning algorithm with the objective of maximizing learning progress. In his experiments, Oudeyer showed that an agent running the IAC algorithm would tend to explore those regions of the state-space that are neither too predictable nor too random, mimicking the intrinsic human drive of curiosity, which continually tries to explore areas that have the highest potential for learning new knowledge. The learning mechanism developed for this thesis was built on the IAC and applied it on an architectural-scale distributed interactive sculptural system. We called it Curiosity-Based Learning Algorithm (CBLA). A CBLA system is comprised of a network of asynchronous learning agents that are linked virtually and physically. The asynchronous nature of the system allows learning agents controlling actuators with different operating bandwidths to coexist. Meanwhile, the links among the different learning agents enable information to travel within the system, allowing distributed learning.

Compared to previous generations of control hardware and software used by the Hylozoic Series \cite{Beesley2010-1}, a CBLA system requires a greatly increased number of sensors and computational power. To maintain a responsive control cycle time, a drastic increase in communication rates also was required. Hence, a new series of electronic hardware and communication architecture with enhanced sensing and communication capabilities was developed to bring CBLA to the Hylozoic Series interactive art sculpture. 

Furthermore, to better understand users' response and perception of the learning behaviour, it is important to be able to capture their responses as they are interacting with the sculpture. An experimental test bed running CBLA was therefore constructed and users were invited to interact with the sculpture. Through this user study, we examined how users perceive and respond to the behaviours of the CBLA system. In particular, the relationship between the behaviours of the sculptural system and the level of interest as reported by the users was investigated. 

\section{Contributions}

First, an algorithm for generating autonomous behaviours for a large scale kinetic sculptural system during interaction with an occupant was developed. A learning algorithm, which was previously used on developmental robotics, was adapted to an interactive arts environment. In \cite{Oudeyer2007}, the state space of the robot was relatively small. However, in our case, there is an order of magnitude more input and output variables. In order to make learning such a large distributed system manageable, a network of multiple learning agents was made to run simultaneously, each capturing a specific set of sensors and actuators. In addition, unlike many reinforcement learning problems, acquiring a model of the learning targets as quickly and accurately as possible isn't our main objective. Instead we want to study the learning process itself and study how curiosity affects the way that the system explores the state-space. The behaviour of the interactive art sculptures is a physical manifestation of the learning process. An additional mechanism that alters the actions of the system based on knowledge gain potential is introduced. This gives the users a visual indication of the state of learning agents. In a way, this ties the users into the learning process and complex behaviours emerge automatically though these interactions. 

Second, a user study was conducted to validate the proposed algorithm. To enable a physical realization of the proposed algorithm, a new prototype interactive control system with enhanced sensing and control capabilities was designed and constructed. The user study examined how human occupants perceive the behaviours of the sculpture and respond in turn. We analyzed the relationships between the behaviours of the system and the users' perceptions. 


\section{Organization}

In \Cref{chap:related_work}, related work in interactive arts, artificial life, and developmental robotics is examined and discussed. In \Cref{chap:cbla}, the proposed Curiosity-Based Learning algorithm is thoroughly described. A new electronic control system was designed and built in order to incorporate the large number of sensors and actuators and to run the CBLA. The design and implementation of the hardware and software of this system are presented in \Cref{chap:ctrl_system}. Then, we observed the behaviours of the sculptural system running the CBLA system under different configurations and examined its interactions with users in a user study. The methodologies and findings of these experiments are presented in \Cref{chap:validations}. Finally, the conclusions as well as ideas for future work are presented in \Cref{chap:conclusions}.
