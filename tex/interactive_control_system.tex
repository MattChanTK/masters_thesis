\chapter[Interactive Control System]
{Interactive Control System\footnote{Part of this chapter is adapted from papers published with IROS~2015~\cite{Chan2015} and Next Generation  \mbox{Building}~\cite{Gorbet2015}}} 
\label{chap:ctrl_system}

To enable the sculpture to understand its own mechanisms, in addition to its surrounding environment and the occupants, there must be at least one proprioceptive sensor associated with each actuator. Note that while the nominal purpose of such a sensor is to directly sense the results of activation, they can also be used to detect changes in the environment, such as interaction with visitors. This implies a large increase in sensing and control capability which the previous versions of the Hylozoic Series embedded electronics \cite{Beesley2010} are unable to provide. A new Hylozoic Series 3 interactive control system was developed with re-designed hardware, to enable the control and sampling of a larger number of actuators and sensors at a higher frequency. In this chapter, we discuss the design of the hardware and software mechanisms that enable the learning algorithm behaviours, and some of the specific actuators and sensors that we use. The design and implementation of the an experimental test bed are also presented. 

\section{System Architecture}

In this implementation, we opted for an architecture where a centralized computer controls a number of smaller localized microcontrollers that interface with the sensors and actuators. By running the high-level algorithm on a central computer, complex algorithms can quickly be developed without the constraints of a embedded platform. In addition, it provides an abstraction layer that enables the designer to build virtual sub-systems using any of the sensors and actuators. These two features allow us to quickly test out different configurations. On the other hand, by distributing the low-level logic on localized microcontrollers, the system can be more modular and the number of wires can be reduced as well as being an important step toward an eventual spatially-distributed CBLA system.. In addition, the real-time nature of the microcontrollers allows protection of sensitive actuators in case of central computer failure. 

\Cref{fig:system-architecture} shows the high-level system architecture of the Hylozoic Series 3 Interactive Control System. Each component runs asynchronously. At the bottom are the Teensy Devices. They represent the physical components in the sculptural system. Each Teensy Device consists of a Teensy Development Board\footnote{Teensy Development Board: \url{www.pjrc.com/teensy/teensy31.html}} that communicates via USB with a computer that hosts the abstract components. Each Teensy controls and samples from a number of actuators and sensors. More details about the physical hardware are presented in \Cref{sec:hardware}. 

\begin{figure} [!htbp]
	\centering
	\includegraphics[height=0.90 \textheight]{"fig/interactive control system/high-level system architecture"}
	\caption[High-level system architecture]{The system is comprised of abstract and physical components. Each component runs asynchronously in its own thread. Each arrow represents the data flow from one component to another.}
	\label{fig:system-architecture}
\end{figure}

For each Teensy Device, a Teensy Interface thread is created with the corresponding communication protocol and system parameters. Each type of Teensy Device has its own unique set of parameters that dictate or reflect the behaviours of its actuators or sensors, respectively. Examples of some of the parameters are the brightness of an LED, and the readings from an accelerometer. The InteractiveCmd module was created to provide a communication layer for all the physical components. Nodes can be created to control or sample any of the actuators and sensors through the InteractiveCmd module. The InteractiveCmd module handles the commands sent from the high-level algorithm and forwards them to the appropriate Teensy Interface in an efficient manner. A Messenger was created to streamline the message delivery system by periodically pushing commands received from the abstract Nodes in a more compressed form. This reduces the number of commands being sent to and received from InteractiveCmd, enabling the system to operate more efficiently.

The Input Nodes and Output Nodes are the abstractions for sensors and actuators. Their sole purpose is to sample or update a particular sensor or actuator continuously. The second level of abstraction are the Device Nodes. Each Device Node is generally made up of one or more Input or Output Nodes, though it can also be completely virtual. It provides additional functionalities such as providing progressive dimming effects to the LEDs, or calculating the running average of a variable. The third level of abstraction are the CBLA Nodes. They are made up of components of the Device Nodes. This is also where the CBLA Engine and Prescripted Engine reside. The internal working of the CBLA Engine is discussed in \Cref{chap:cbla}. A Prescripted Engine is an alternate set of behaviours that follow user designed scripts, and do not involve any learning algorithm. A CBLA Node can be switched to either Engine while it is running. 

The Data Logger is a special kind of abstract component that periodically samples the abstract variables of each abstract node, packages the data in the memory, and saves data in the hard drive in an efficient manner. The data collected are stored in a persistent key-value database for further analysis. In addition, the entire state of the CBLA Engine is also saved. This allows the CBLA system to recover from a previous state at a later time.


\section{Hardware} \label{sec:hardware}

The hardware was custom designed in collaboration with Mohammadreza Memarian. The goal was to develop a system that is modular, flexible, and expandable. Analogue sensors and actuators that were traditionally used in previous Hylozoic Series as well as off-the-shelf sensors that interface via I2C, SPI, or UART are supported. Most importantly, the design enables high-speed two-way communication with a computer over USB. Sensor readings and control signals can be delivered to and from a computer at around 100Hz. This capability allows us to run algorithms that are more computationally intensive; simplify the implementation of multi-threaded and multi-process software; and collect and display a large amount of data at runtime on a standard computer. 

\subsection{Actuators and Sensors}

The sensors and actuators used in the experiments described in \Cref{chap:validations} and their interface types are tabulated in \Cref{table:sensors-list} and \Cref{table:actuators-list} respectively.

\begin{table}[!htb]
	\caption[List of sensors and their interface types]{List of sensors that were used in the experiments and their interface types}
	\begin{center}
		\begin{tabular}{ | c | c |} 
			\hline
			\textbf{Sensor} & \textbf{Interface Type} \\ 
			\hline
			IR proximity sensor\tablefootnote{Sharp GP2Y0A21YK Infrared Proximity Sensor:  \url{www.sharpsma.com/webfm_send/1208}} & Analogue  \\ 
			\hline
			Accelerometer\tablefootnote{ADXL345 3-Axis Digital Accelerometer: \url{	www.analog.com/media/en/technical-documentation/data-sheets/ADXL345.PDF}} & I2C \\
			\hline
			Ambient light sensor\tablefootnote{SparkFun Ambient Light Sensor Breakout (TEMT6000):  \url{www.sparkfun.com/products/8688}} & Analogue \\ 
			\hline
		\end{tabular}
	\end{center}
	\label{table:sensors-list}
\end{table}


\begin{table}[!htb]
	\caption[List of actuators and their interface types]{List of actuators that were used in the experiments and their interface types}
	\begin{center}
		\begin{tabular}{ | c | c |} 
			\hline
			\textbf{Actuator} & \textbf{Interface Type} \\ 
			\hline
			Shape memory alloy (SMA) wire\tablefootnote{Dynalloy Flexinol Actuator Wire:  \url{www.dynalloy.com/tech_data_wire.php}} & PWM  \\ 
			\hline
			Vibration motor & PWM \\ 
			\hline			
			LED & PWM \\ 
			\hline		
			High-power LED\tablefootnote{Indus Star High-Power LED Light Module:  \url{www.luxdrive.com/content/A007_A008_Data_Sheet_V1.2.pdf}} & PWM \\ 
			\hline
		\end{tabular}
	\end{center}
	\label{table:actuators-list}
\end{table}

The aforementioned actuators and sensors interface with the Control Module through some custom drivers called Device Modules. Different actuators or sensors require different Device Modules due to their different power requirements. They are plugged into the Device Ports on the Control Modules (discussed in \Cref{subsec:controller}). These Device Modules provide the power switching and voltage regulation needed to drive the actuators and sensors. 

Combinations of the actuators and sensors listed in \Cref{table:sensors-list,table:actuators-list} are used to form three functional units: 1) Fin Unit, which is composed of two SMA wires, one infrared (IR) proximity sensor, and a three-axis accelerometer; 2) Reflex Unit, which has a pair of LEDs (wired in parallel), a vibration motor, and one infrared (IR) proximity sensors; and 3) Light unit, which comprises a high-power LED and an ambient light sensor. Their spatial configurations are shown in \Cref{fig:Physical hardware}.

\begin{figure}[!htb]
	\centering
	\includegraphics[height=0.85 \textheight]{"fig/interactive control system/Physical hardware"}
	\caption[Typical spatial configuration of the physical functional units]{This is a typical spatial configuration of the functional units (figure adapted from an image provided by Philip Beesley Architect Inc.). Bright green denotes actuators and bright red denotes sensors. Light Unit is shaded yellow (1 shown); Fin Unit is shaded green (2 shown); and Reflex Unit is shaded purple (2 shown).}
	\label{fig:Physical hardware}
\end{figure}

A Fin Unit moves a 2-DOF Fin-like mechanism made of flexible plastic rods, by controlling its two SMA wires. A Reflex Unit actuates a vibration motor and mobilizes the Frond-like mechanism attached to it. The pair of LEDs in the Reflex Unit shine on to the Frond for additional effect. A Light Unit contains a very bright LED and it is hung in the canopy of the sculpture, above the visitors. The IR proximity sensors detect the position of the occupants or objects within their ranges. The accelerometers detect contacts with the occupants and the deformations of the Fins. The ambient light sensors detect the light intensities of the ambient environment and their associated LEDs. 


\FloatBarrier
\section{Control Software}

The control software of the sculpture consists of a low-level firmware layer and high-level application layer. The two layers are connected via USB. The low-level layer consists of the firmware that interfaces with the peripherals that connect with the actuators and sensors. It resides on a custom Control Module that interfaces with the physical electronic components. The implementation details of the Control Module can be found in \Cref{Append:Control Module}. The high-level layer consists of the tools that facilitate communication between the two layers and the application that dictates the system behaviour. The abstraction provided by the high-level layer allows flexibility in defining the nodes and their relationships to each other. 

\subsection{Communication and Interface}

A low-level layer of firmware written in C++ runs on the Teensy 3 USB-based development boards which interface with the peripherals that connect with the actuators and sensors. High-level software written in Python 3.4 runs on a central computer. The use of the central computer as a development platform provides flexibility for development free from the limited processing power and specialized functions inherent to the Teensy microcontroller hardware. Moreover, Python 3.4 is cross-platform and supports multi-threading, permitting operation within many operating systems and allowing multiple sets of software instructions to be executed in parallel. Code that is necessary for communicating with the low-level layer is packed into a Python Package named \textit{interactive-system}. Developers can then develop applications that control and retrieve information from the sculptural system firmware using the software utilities provided by the Python Package. Each application can run on its own thread. While care should be exercised to avoid conflicts among threads, this should permit multiple applications to execute simultaneously.	

CBLA executes as an application that communicates with the low-level layer using the \textit{interactive-system} Python package. Other applications such as an occupancy map that uses the sensors on the sculpture to interpret the locations of the occupants can run simultaneously, taking advantage of the multi-threading properties of the high-level platform.

\Cref{fig:TeensyInterface Thread} illustrates how an application communicates with the Teensy devices. At the high-level layer, the Teensy Interface module in the \textit{interactive-system} package is used to create a thread for each Teensy device. The thread looks for changes in those parameters and performs synchronization. Each Teensy device on the low-level layer is represented by a Teensy Interface thread on the high-level layer. Teensy devices are considered as slave devices in this communication mechanism. Only the Teensy Interface, the Master, can initiate a read or write request. An InteractiveCmd thread can modify a Teensy's output parameters and retrieve its input parameters through its Teensy Interface. 

\begin{figure}[!htbp]
	\centering
	\includegraphics[width=0.9 \textwidth]{"fig/interactive control system/TeensyInterface Thread"}
	\caption[Illustration of the working of the Teensy Interface]{Teensy Interface is the connection between the InteractiveCmd and its associated Teensy device. InteractiveCmd modifies output parameters on its corresponding Teensy Interface's output hash table. This triggers an event that pushes the changes to the Teensy device. The Teensy device would then respond by triggering an event that updates the input hash table with newly sampled input values and notifies the InteractiveCmd.}
	\label{fig:TeensyInterface Thread}
\end{figure}

\FloatBarrier
\subsection{Node Abstraction}

Between the Nodes and the Teensy Interface, there is the InteractiveCmd. Its job is to forward messages to the correct Teensy Interface and hide the physical implementation of the low-level layer devices from the Nodes. In addition, since the InteractiveCmd module enables the control and sampling of any actuators and sensors in the system, a Node can be constructed unconstrained by spatial or hardware specificities. Each Node, physical or virtual, is represented by a set of input and output variables which can be accessed by any other nodes in the system, and each runs continuously in its own thread. Input variables are simply variables in the memory that Nodes have read access to. Similarly, output variables are variables that Nodes have write access to. Multiple Nodes can share one input variable while only one Node can be associated with one output variable. \Cref{fig:Node_abstraction} illustrates the relationships between different Nodes. 

\begin{figure}[!htbp]
	\centering
	\includegraphics[width=1.0 \textwidth]{"fig/interactive control system/Node_abstraction"}
	\caption[Illustration of how abstract nodes work]{Illustration of how abstract nodes work. Each rectangle represents a thread. The purple one represents InteractiveCmd module and each green one represents a Node. The blue circle represents a variable in the memory. The arrow represents the direction of access. Multiple Nodes can share read access to a variable while only one Node can have write access to one variable.}
	\label{fig:Node_abstraction}
\end{figure}

At the lowest level, Input Nodes continuously update their associated variables, and Output Nodes continuously send output requests to InteractiveCmd through the Messenger. Different types of Input and Output nodes are configured to run at a loop period compatible with the physical components that they represent. This mechanism makes implementation of the higher level Nodes much easier by eliminating the need for communicating with the InteractiveCmd by means of sending individual messages. Instead, each of the input and output variables can be accessed from the memory at any time. These variables are used as building blocks for higher level Nodes. In addition, intermediate level Nodes can embed extra functionalities. For instance, a LED Driver ramps up or dims an LED to the desired brightness level. A higher-level Node controlling that LED using the LED Driver can then operate at a lower update period and process more complex logic. This Node Abstraction system makes developing CBLA Nodes much simpler by eliminating the need for managing logic requiring different frequencies of control under one thread. 

The addition of the Messenger node between the InteractiveCmd, and Input and Output Nodes streamlines the communication by reducing the number of messages. Over USB, each packet can contain up to 64 bytes. If each Node communicates with the InteractiveCmd directly, there will be many messages that might only require one or two bytes. A large portion of the communication bandwidth will be wasted and the update rate of the Nodes will be significantly throttled. Since many messages are likely to be delivered to the same Teensy, those messages can be combined and delivered as a single packet. The job of the Messenger Node is to collect all the messages, combine them appropriately, and deliver them to the InteractiveCmd periodically. Although this means that each message must wait for the next delivery cycle to be sent out, this mechanism allows the system to handle a much higher throughput. To avoid commands or requests being missed, the rate of each Input or Output node is set to be at least three times the Messenger's update period. 


\FloatBarrier
\section{Data Logging}

For secondary analysis, the values of all input and output variables of every Node, as well as the internal variables within each CBLA Engine must be collected and stored on the hard drive. In addition, the state of the CBLA Learner, including all the exemplars, the prediction models, and the definitions of the regions must be stored such that it can be recovered at a later time. 

The CBLA system contains a large number of asynchronous threads that run at their own speeds. As a result, a large amount and variety of data are generated at high frequencies and at different times. These data must be handled in a way that does not slow down the system. In addition, in case the program fails to terminate safely, the majority of the data should still be recoverable. Also, the data must be saved to disk and be discarded from the memory continuously as the CBLA system is expected to operate for a long period of time. A specialized Data Logging module was developed to facilitate the collections of the high volume of data needed for validation purposes. Details about the implementations can be found in \Cref{Append:Data Logging}.



\section{Multi-Cluster Test Bed}\label{sec:multi-cluster-test-bed}

An experimental test bed was built to investigate how users interact with the CBLA system. This is a small scale four-cluster test bed that resembles a typical interactive sculpture produced by Philip Beesley Architect Inc. (PBAI) and it was used in the experiments described in \Cref{sec:multi-cluster-experiment,sec:user-study}. A photograph of the complete test bed is shown in \Cref{fig:cbla-test-bed photo 2}

\begin{figure} [!htb]
	\centering
	\includegraphics[width=1.0\textwidth]{"fig/validations/cbla-test-bed photo 2"}
	\caption[Photograph of the multi-cluster test bed]{Photograph of the multi-cluster test bed when actuated. The names of the Clusters from left to right are: Cluster 1 ($C1$), Cluster 2 ($C2$), Cluster 3 ($C3$), and Cluster 4 ($C4$).}
	\label{fig:cbla-test-bed photo 2}
\end{figure}


\subsection{Electronic Components}

A Light Unit is made up of one high-power LED and one ambient light sensor (shown in \Cref{fig:Light Unit}). The high-power LED is mounted on top of a flask containing coloured liquid. The ambient light sensor is mounted on the side of the flask under the LED. This allows the ambient light sensor to measure the intensity of the light emitted by the LED. 

A Fin Unit is made up of two SMA wires, a pair of LEDs, a vibration motor, a 3-axis accelerometer, and two IR proximity sensors. The two SMA wires pull on two levers that move a Fin, which is a mechanism made of soft plastic rods that curls up (shown in \Cref{fig:Fin Unit}). An IR proximity sensor and an accelerometer are mounted at around midway between the tip to the root of the Fin. At the bottom of the Fin, a vibration motor, a pair of LEDs, and an IR proximity sensor are mounted in the middle of two frond-like objects (shown in \Cref{fig:Reflex Unit}). 

\begin{figure}[!htb]
	\centering
	\begin{tabular}[t]{cc}
		\begin{tabular}{c}
			\begin{subfigure}{0.40\textwidth}
				\centering
				\includegraphics[width=0.99\textwidth]{"fig/validations/Light Unit"}
				\caption{Light Unit}
				\label{fig:Light Unit}
			\end{subfigure}	
		\end{tabular}
		&
		\begin{tabular}{c}
			\begin{subfigure}{0.38\textwidth}
				\centering
				\includegraphics[width=0.99 \textwidth]{"fig/validations/Fin Unit"}
				\caption{Fin Unit (top)}
				\label{fig:Fin Unit}
			\end{subfigure}	
		 	\\
			\begin{subfigure}{0.38\textwidth}
				\centering
				\includegraphics[width=0.99 \textwidth]{"fig/validations/Reflex Unit"}
				\caption{Fin Unit (bottom)}
				\label{fig:Reflex Unit}
			\end{subfigure}
		\end{tabular}
	\end{tabular}
	\caption[Photograph of components in the multi-cluster test bed]{Photograph of components in the multi-cluster test bed.}
	\label{fig:cbla-test-bed components}
\end{figure}



\subsection{Device Nodes}

Device Nodes further abstract the Output and Input Nodes to enable higher level functionality. This frees the CBLA Nodes from managing the constraints imposed by the physical attributes of the actuators and sensors. 

\subsubsection{SMA Controller Node}

In the experiment described in \Cref{sec:multi-node}, the SMA wires were only operated in fully-off or fully-on mode. This means that a Fin with two SMA wires can only have four possible states. In addition, each actuation must be a cycle since the SMA wires cannot be fully actuated at 5V for more than 2 seconds. Empirically, we determined that the cooling period takes around 10 seconds. This means that the loop period for a CBLA Node cannot be lower than 12 seconds since it does not have the freedom to actuate the SMA wires again during the cooling period. These restrictions are problematic during interaction with the users since the learning period and response latency would likely take longer than what a typical visitor would spend in front of a section of a sculpture. In addition, only a very coarse model can be made with only four possible actions that the Fin can choose from. This means that the Fin Node would likely be very unresponsive since the variance in resultant state for each kind of action is likely to be very high.

Thus, for this experiment, a position controller Node is needed to enable the SMA wire to hold its contraction while keeping the SMA wire at a safe temperature. Since the length of an SMA wire is related to the temperature, and current is passed through the SMA wire to generate heat, to maintain a position, the controller needs to adjust the output voltage to a level that can maintain the desired temperature. This can reduce the loop period as the SMA wires no longer need to cool down after actuating. In addition, the number of actions is no longer limited to four as the Node can select any value between fully-off and fully-on. However, it is difficult to attach a temperature sensor on the SMA wire. Thus, instead of using a feedback controller, only an open-loop controller with a model that estimates the temperature of an SMA wire is used.

This controller essentially produces a control signal that allows the SMA wire to quickly reach its desired temperature by setting the output voltage very high. It then gradually lowers the voltage as the SMA wire reaches its desired position according to an internal model. The implementation of this SMA Controller Node can be found in \Cref{Append:SMA Controller}. Due to the lack of feedback control and the simplifications and approximations made when developing the model, this controller is unlikely to be accurate. However, the main purpose of this SMA Controller is to enable the CBLA Node to hold the Fin at a particular position. As a result, this allows the CBLA Engine to run at a higher rate and expands the number of possible actions. 


\subsubsection{LED Driver Node}

If a CBLA Node controls an LED directly, any change in brightness would be set instantly. As a result, the LED may appear to be flicking or flashing erratically to the viewers as it rapidly jumps between different brightness levels. To improve the aesthetic, an LED Driver that brightens and dims the LED gradually was developed. It allows a Node to set a particular target brightness level and the LED driver would increment or decrement the brightness of the LED linearly by adjusting the output PWM voltage until it reaches the target. Its design details can be found in \Cref{Append:LED Driver}

  
\subsection{Isolated CBLA Nodes}\label{sec:isolated-cbla-node}

Each Isolated CBLA Node is associated with one actuator. A CBLA system, as discussed in \Cref{sec:cbla-system}, is constructed by linking these Isolated CBLA Nodes through virtual inputs. In this test bed, three main types of Isolated CBLA Node exist: Half-Fin Node, Light Node, and Reflex Node.

\Cref{fig:Isolated CBLA Nodes} presents the make-up of the different Isolated CBLA Nodes in a cluster. Two Half-Fin Nodes control the bending of a Fin through their respective SMA Controller Nodes (Fx.SMA-L and Fx.SMA-R). The pair of Half-Fin Nodes share a Fin-mounted IR proximity sensor (Fx.IR-F), and the 3 axes of the accelerometer (Fx.ACC). A Light Node controls the brightness of a high-power LED through an LED driver Node (Lx.LED) and its sensory space consists of an ambient light sensor (Lx.ALS). There are two types of Reflex Nodes, one is associated with a pair of LEDs (Fx.RFX-L), and one is associated with a vibration motor (Fx.RFX-M). They are both controlled through LED Driver Nodes. In their sensory space, they share one bottom-mounted IR proximity sensor (Fx.IR-S). 

\begin{figure} [!htbp]
	\centering
	\includegraphics[width=1.0\textwidth]{"fig/validations/Isolated CBLA Nodes"}
	\caption[Make-up of a cluster of Isolated CBLA Nodes]{Make-up of a cluster of Isolated CBLA Nodes. Half-Fin Nodes are shown in red; Light Nodes are shown in orange; and Reflex Nodes are shown in blue.}
	\label{fig:Isolated CBLA Nodes}
\end{figure}


\subsection{Network Configurations}

The Isolated CBLA Nodes described in \Cref{sec:isolated-cbla-node} are connected to each other via virtual inputs. In essence, the output of a CBLA Node is treated as an input variable to other Nodes, much like input variables associated with their own sensors. This creates a interconnected network of CBLA Nodes that spans across the entire sculptural system. This allows information regarding the external environment to travel through the sculpture. 

Different network configurations produce different system behaviours. Here we investigate two types of network configurations which are called Spatial Mode and Random Mode. For fair comparisons, the two configurations have exactly the same number of links. In the experiment, the number of links was arbitrarily chosen to be 111 so that there are sufficient links to cover the entire system. In addition, each Node is linked to at least one other Node via virtual input. 

\subsubsection{Spatial Mode}

In Spatial Mode, CBLA Nodes are linked based on their spatial proximity. \Cref{fig:Spatial Local Modes} shows the connections among Nodes within a cluster. Neighbouring Nodes are connected via bidirectional links. Two connected Nodes take each other's output as input variables. Nodes in neighbouring clusters that are in close proximity are linked via unidirectional links as shown in \Cref{fig:Spatial Global Mode}. The circular connections that daisy-chain the four clusters allow information to spread throughout the test bed. 

\begin{figure} [!htbp]
	\centering
	\includegraphics[width=1.0\textwidth]{"fig/validations/Spatial Local Mode"}
	\caption[Connectivity graph within a cluster in Spatial Mode]{Connectivity graph within a cluster in Spatial Mode. Each bidirectional green arrow represents a pair of input links. Two connected Nodes take each other's output as input variable. Note that each bidirectional link is counted as two links.}
	\label{fig:Spatial Local Modes}
\end{figure}

\begin{figure} [!htbp]
	\centering
	\includegraphics[height=0.85 \textheight]{"fig/validations/Spatial Global Mode"}
	\caption[Connectivity graph of the entire test bed in Spatial Mode]{Connectivity graph of the entire test bed in Spatial Mode. Each blue unidirectional arrow represents an input link. The output of the Node at the origin of an arrow is fed into the Node that it is pointing toward as an input variable.}
	\label{fig:Spatial Global Mode}
\end{figure}

Under this configuration, Nodes' responses to external disturbances or environmental changes are expected to concentrate close to the source, and spread to other Nodes gradually. 

\subsubsection{Random Mode}

In Random Mode, CBLA Nodes are linked to other CBLA Nodes randomly. This means that the output of a Node might be fed into another Node that is on the other side of the sculpture. Since there are 60 Nodes and 111 links are required, the algorithm first links the output of each Node to another random Node to ensure that each Node has an effect in the overall system. Then, the algorithm picks 51 unique Nodes at random and links each of them to another random Node. This ensures that the state space of each Node is relatively even. 

The specific random configuration is generated at run-time and is different every time. In addition, it should be noted that this method does not guarantee that there will not be any disconnected sub-networks. It is possible that a set of Nodes is completely isolated from another set of Nodes under this network configuration scheme.

\subsection{Prescripted Behaviours}\label{sec:prescripted-behaviours}

For the purpose of comparing between CBLA and prescripted behaviours, each CBLA Node has a Prescripted Engine in addition to the CBLA Engine. This allows us to quickly switch between the two kinds of behaviours during the user study described in \Cref{sec:user-study}. Although the two engines are both associated with the same actuators, they may have different sensors in its sensory space. 

The prescripted behaviours are implemented based on a specification by a human designer, and are similar to previous behaviours of the Hylozoic series\cite{Gorbet2015}. For the Fin mechanism, when its Fin-mounted IR proximity sensor detects an object, it bends down toward the direction of a neighbouring bottom-mounted IR proximity sensor that has also detected an object. If both or neither of IR proximity sensors have detected an object, it simply bends straight down. It returns to an upright rest position when its Fin-mounted IR proximity sensor no longer detects an object in its proximity.

For the high-power LEDs, its output ramps up and down continuously when its corresponding Fin-mounted IR proximity sensor has detected an object. It then dims gradually when the object is removed.

For the reflex vibration motor or LED pair, its output also ramps up and down continuously when its corresponding bottom-mounted IR proximity sensor has detected an object. It then ramps down gradually when the object is removed.

An additional virtual node is added to provide cluster-level group behaviours. This node counts the number of outputs within its cluster that are active. It then determines a probability of random activation by mapping this count to a Gaussian function. The Fin mechanism or the high-power LED may turn on at random based on this probability. Using a Gaussian function allows the probability of random activation to increase when a number of outputs are activated. However, when too many outputs are activated, this probability decreases which makes random activations less probable. 

The prescripted behaviours described above was designed to be highly responsive, along with elements of group behaviours and random actuations. These kinds of behaviours are more similar to the types of behaviours of a typical interactive sculpture produced by LASG and PBAI than the CBLA-based behaviours introduced in this thesis. Its main purpose was to serve as a reference point when the visitors' responses to the CBLA system was studied in the user study described in \ref{sec:user-study}.

%TODO probably should add a summary