% T I T L E   P A G E
% -------------------
% Last updated May 24, 2011, by Stephen Carr, IST-Client Services
% The title page is counted as page `i' but we need to suppress the
% page number.  We also don't want any headers or footers.
\pagestyle{empty}
\pagenumbering{roman}

% The contents of the title page are specified in the "titlepage"
% environment.
\begin{titlepage}
        \begin{center}
        \vspace*{1.0cm}

        \Huge
        {\bf Curiosity-Based Learning Algorithm for Interactive Art Sculptures}

        \vspace*{1.0cm}

        \normalsize
        by \\

        \vspace*{1.0cm}

        \Large
        Matthew Tsz Kiu Chan \\

        \vspace*{3.0cm}

        \normalsize
        A thesis \\
        presented to the University of Waterloo \\ 
        in fulfillment of the \\
        thesis requirement for the degree of \\
        Master of Applied Science \\
        in \\
        Electrical and Computer Engineering \\

        \vspace*{2.0cm}

        Waterloo, Ontario, Canada, 2016 \\

        \vspace*{1.0cm}

        \copyright\ Matthew Tsz Kiu Chan 2016 \\
        \end{center}
\end{titlepage}

% The rest of the front pages should contain no headers and be numbered using Roman numerals starting with `ii'
\pagestyle{plain}
\setcounter{page}{2}

\cleardoublepage % Ends the current page and causes all figures and tables that have so far appeared in the input to be printed.
% In a two-sided printing style, it also makes the next page a right-hand (odd-numbered) page, producing a blank page if necessary.
 


% D E C L A R A T I O N   P A G E
% -------------------------------
  % The following is the sample Delaration Page as provided by the GSO
  % December 13th, 2006.  It is designed for an electronic thesis.
  \noindent
I hereby declare that I am the sole author of this thesis. This is a true copy of the thesis, including any required final revisions, as accepted by my examiners.

  \bigskip
  
  \noindent
I understand that my thesis may be made electronically available to the public.

\cleardoublepage
%\newpage

% A B S T R A C T
% ---------------

\begin{center}\textbf{Abstract}\end{center}

This thesis is part of the research activities of the Living Architecture System Group (LASG). Combining techniques in architecture, the arts, electronics, and software, LASG develops interactive art sculptures that engage occupants in an immersive environment. The overarching goal of this research is to develop architectural systems that possess life-like qualities. Recent advances in miniaturization of computing and sensing units enable system-wide responsive behaviours. Though complexity may emerge in current LASG systems through superposition of a set of simple and prescripted behaviours, the responses of the systems to occupants remain rather robotic and ultimately dictated by the will of the designers. In this thesis, a new series of sculptural system was initiated, implementing an additional layer of behavioural autonomy.  

In this thesis, the Curiosity-Based Learning Algorithm (CBLA), a reinforcement learning algorithm which selects actions that lead to maximum potential knowledge gains, is introduced to enable the sculpture to automatically generate interactive behaviours and adapt to changes. The CBLA allows the sculptural system to construct models of its own mechanisms and its surroundings through self-experimentation and interaction with human occupants. A novel formulation using multiple learning agents, each comprising a subset of the system, was developed in order to integrate a large number of sensors and actuators. These agents form a network of independent, asynchronous CBLA Nodes that share information about localized events through shared sensors and virtual inputs. Given different network configurations of the CBLA system, the emergence of system behaviours with varying activation patterns was observed. 

To realize the CBLA system on a physical interactive art sculpture, an overhaul of the previous series' interactive control hardware was necessary. CBLA requires the system to be able to sense the consequences of its own actions and its surrounding at a much higher resolution and frequency than prior implemented behaviour algorithms. This translates to the need to interface and collect samples from a substantially larger number of sensors. A new series of hardware as well as control system software was developed, which enables the control and sampling of hundreds of devices on a centralized computer through USB connections. Moving the computation from an embedded platform simplifies the implementation of the CBLA system, which is a computationally intensive and complex program. In addition, the large amount of data generated by the system can now be recorded without sacrificing response time nor resolution.  

An experimental test bed was built to validate the behaviours of the CBLA system. This small-scale interactive art sculpture resembles previous sculptures displayed publicly by the LASG and Philip Beesley Architect Inc (PBAI). Experiments were done on the testbed at PBAI's Toronto studios, to demonstrate the exploratory patterns of CBLA as well as the collective learning behaviours produced by the CBLA system. Furthermore, a user study was conducted to better understand users' responses to this new form of interactive behaviour. Comparing with prescripted behaviours that were explicitly programmed, the participants of the study did not find this implementation of the CBLA system more interesting. However, the positive correlations between activation level, responsiveness, and users' interest levels revealed insights about users' preferences and perceptions of the system. In addition, observations during the trials and the responses from the questionnaires showed a wide variety of user behaviours and expectations. This suggests that, in future work, results should be categorized to analyze how different types of users respond to the sculpture. Moreover, the experiments should also be designed to better reflect the actual use cases of the sculpture. 



\cleardoublepage
%\newpage

% A C K N O W L E D G E M E N T S
% -------------------------------

\begin{center}\textbf{Acknowledgements}\end{center}

I would like to thank all the people who made this possible.
\cleardoublepage
%\newpage

% D E D I C A T I O N
% -------------------

\begin{center}\textbf{Dedication}\end{center}


\cleardoublepage
%\newpage

% T A B L E   O F   C O N T E N T S
% ---------------------------------
\renewcommand\contentsname{Table of Contents}
\tableofcontents
\cleardoublepage
\phantomsection
%\newpage

% L I S T   O F   T A B L E S
% ---------------------------
\addcontentsline{toc}{chapter}{List of Tables}
\listoftables
\cleardoublepage
\phantomsection		% allows ref to link to the correct page
%\newpage

% L I S T   O F   F I G U R E S
% -----------------------------
\addcontentsline{toc}{chapter}{List of Figures}
\listoffigures
\cleardoublepage
\phantomsection		% allows hyperref to link to the correct page
%\newpage

% L I S T   O F   S Y M B O L S
% -----------------------------
% To include a Nomenclature section
% \addcontentsline{toc}{chapter}{\textbf{Nomenclature}}
% \renewcommand{\nomname}{Nomenclature}
% \printglossary
% \cleardoublepage
% \phantomsection % allows hyperref to link to the correct page
% \newpage

% Change page numbering back to Arabic numerals
\pagenumbering{arabic}

