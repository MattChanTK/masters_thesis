\chapter{Curiosity-Based Learning Algorithm} \label{chap:cbla}
 The CBLA functions by exploiting the system’s inherent curiosity to learn about itself, much like an infant might learn by exercising groups of muscles and observing the response.  In its simplest form, the algorithm chooses an action from its action repertoire to perform, and measures the response.  At the same time, it generates a prediction of what it thinks should happen.  If the prediction matches the measured response, it has learned that part of its sensorimotor space and that space becomes less interesting for future actions.  If the prediction fails to match the measured response, it remains curious about that part of its “self,” as it obviously still has more to learn.  It will create a new prediction and try again.  This learning architecture allows the system to learn both about itself, and also about interactions with occupants, whose movements and actions create new and “surprising” responses, activating the system’s curiosity.

\section{Adaptation from Oudeyer’s Intrinsic Adaptive Curiosity}

\section{CBLA Engine}

\section{Multi-Node CBLA System}

