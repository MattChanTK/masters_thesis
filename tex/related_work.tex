\chapter{Related Work} \label{chap:related_work}

\section{Interactive Arts}
Interactive arts can be categorized as Static, Dynamic-Passive, Dynamic-Interactive, and Dynamic-Interactive (varying) based on the degree of the interaction between the art works and the viewers \cite{Edmonds2004}. Dynamic-Interactive systems give the human viewers an active role in defining the behaviours of the system. This category introduces an agent that modifies the specifications of the art object. This additional unpredictability introduces a new dimension of complexity to the behaviours of the system. 
In \cite{Drummond2009}, Drummond examined the conceptual elements of interactive musical arts. For an interactive system to move beyond being simply a complex musical instrument with a direct reactive mapping from inputs to generation of sound, it must possess some level of autonomy in the composition and generation of music. In addition, the interactivity should be bilateral; the performer influences the music and the music influences the performer. These concepts can easily be extended to visual and kinetic arts. Visual-based interactive systems such as the Iamascope in Beta\_space \cite{Costello2005} and audio-based systems such as Variations \cite{Wands2005} engaged the participants by allowing them to directly affect the output of the system. Works in interactive architecture [1] [8] try to provide a responsive and immersive environment where the viewers can feel as if they belong to the system. 
However, most of these works are the non-varying type of Dynamic-Interactive system, as their responsive behaviours do not change. Over a longer term, the system will become more predictable and its effectiveness in engaging the users will consequently decrease. In this work, we aim to create a varying interactive artwork by emulating the characteristics of living organisms such as curiosity and learning \cite{Beesley2012.book}. 

\section{Artificial Life}

To emulate life-like behaviours, one can start by observing how human beings behave. \cite{Dragan2015}, \cite{Dragan2014} modelled how human beings convey or mask their intentions through movement and applied these models on a humanoid robot. Similarly, \cite{Gielniak2013} focuses on making the robot’s motion more understandable by emulating the coordinated effects of human joints. Those studies focus their attention on making the intent of the robots clear. In contrast, our objective is to make robots more engaging and life-like, where unpredictability might be a desirable quality. For instance, \cite{Dragan2014} showed that the robot’s perceived intelligence increased when the participants believed that the robot was intentionally deceptive. Our work investigates whether unpredictable behaviours emerging from the learning process will appear more life-like and engaging.
One of the open questions in artificial life research is whether we can demonstrate the emergence of intelligence and mind \cite{Bedau2000}, examined in projects such as the Petting Zoo by Minimaforms \cite{Minimaforms} and Mind Time Machine \cite{Ikegami2013}.   The idea of emergence of structure and consciousness is explored in many previous works in the field of developmental robotics \cite{Lungarella2003} \cite{Asada2009} \cite{Kompella2014}.  Oudeyer et al. developed a learning mechanism called Intelligent Adaptive Curiosity (IAC) \cite{Oudeyer2007}, a reinforcement learning algorithm with the objective of maximizing learning progress. In his experiments, he showed that an agent would tend to explore state-space that is neither too predictable nor too random, mimicking the intrinsic human drive of curiosity, which continually tries to explore areas that have the highest potential for learning new knowledge. However, previous work did not cover how the IAC might be scaled to a distributed system with a large sensorimotor space. 

\section{Developmental Robotics}


\section{Machine Learning}

\section{User Experience Survey}
